% !TEX TS-program = pdflatex
% !TEX encoding = UTF-8 Unicode

% This is a simple template for a LaTeX document using the "article" class.
% See "book", "report", "letter" for other types of document.

\documentclass[11pt]{scrartcl} % use larger type; default would be 10pt

\usepackage[utf8]{inputenc} % set input encoding (not needed with XeLaTeX)

%%% Examples of Article customizations
% These packages are optional, depending whether you want the features they provide.
% See the LaTeX Companion or other references for full information.

%%% PAGE DIMENSIONS
\usepackage{geometry} % to change the page dimensions
\geometry{a4paper} % or letterpaper (US) or a5paper or....
% \geometry{margin=2in} % for example, change the margins to 2 inches all round
% \geometry{landscape} % set up the page for landscape
%   read geometry.pdf for detailed page layout information

\usepackage{graphicx} % support the \includegraphics command and options

% \usepackage[parfill]{parskip} % Activate to begin paragraphs with an empty line rather than an indent

%%% PACKAGES
\usepackage{booktabs} % for much better looking tables
\usepackage{array} % for better arrays (eg matrices) in maths
\usepackage{paralist} % very flexible & customisable lists (eg. enumerate/itemize, etc.)
\usepackage{verbatim} % adds environment for commenting out blocks of text & for better verbatim
\usepackage{subfig} % make it possible to include more than one captioned figure/table in a single float
\usepackage{float}
% These packages are all incorporated in the memoir class to one degree or another...

%%% HEADERS & FOOTERS
\usepackage{fancyhdr} % This should be set AFTER setting up the page geometry
\pagestyle{fancy} % options: empty , plain , fancy
\renewcommand{\headrulewidth}{0pt} % customise the layout...
\lhead{}\chead{}\rhead{}\lhead{}\chead{\textit{CS27020 Assignment: Ski Lifts and Pistes : jee22}}\rhead{}
\lfoot{\textit{Aberystwyth University}}\cfoot{\thepage}\rfoot{\textit{Computer Science}}

%%% SECTION TITLE APPEARANCE
\usepackage{sectsty}
\allsectionsfont{\sffamily\mdseries\upshape} % (See the fntguide.pdf for font help)
% (This matches ConTeXt defaults)

%%% ENVIRONMENT ITEMIZE
\newenvironment{myitemize}
{ \begin{itemize}[]
    \setlength{\itemsep}{0pt}
    \setlength{\parskip}{0pt}
    \setlength{\parsep}{0pt}     }
{ \end{itemize}                  } 

%%% ToC (table of contents) APPEARANCE
\usepackage[nottoc,notlof,notlot]{tocbibind} % Put the bibliography in the ToC
\usepackage[titles,subfigure]{tocloft} % Alter the style of the Table of Contents
\renewcommand{\cftsecfont}{\rmfamily\mdseries\upshape}
\renewcommand{\cftsecpagefont}{\rmfamily\mdseries\upshape} % No bold!

%%% END Article customizations

%%% The "real" document content comes below...

\title{Brief Article}
\author{The Author}
%\date{} % Activate to display a given date or no date (if empty),
         % otherwise the current date is printed 

\begin{document}
\newpage

\begin{center}
\textbf{\LARGE CS27020}\\[0.5cm]

\textbf{\LARGE Assignment: Ski Lifts and Pistes }\\[13cm]

\begin{tabular}{ l | r }
{\large Author:} & {\large Address:} \\
James Euesden (jee22) &  Department of Computer Science \\
 & Aberystwyth University \\ 
 & Aberystwyth \\
 & Ceredigion \\
 & SY23 3DB \\
{\large Date:} \today  &  \\

\end{tabular} \\[0.2cm]

{\small Copyright © Aberystwyth University 2014}

\end{center}

%%%\newpage

%%%\tableofcontents

\newpage

\section{Introduction}
%%COMPOSE
\section{Analysis}
\subsection{Functional Dependencies}
To see the functional dependencies, I made a number of assumptions about the data, based upon the sample data provided.
\subsubsection{Piste}
 \textbf{pisteName -\textgreater  grade, length, fall, open}
\\[0.1cm]
%% REWRITE
pisteName does not functionally determine the liftName, as liftName could be a number of things. It does, however, determine the grade, length, fall and whether or not it is open. A logical primary key, based on the assumption that no Piste will be named the same, would be piesteName. This must be brought into First Normal Form (1NF) in order for this to work though.\\
By using pisteName, we are assuming that no 2 Pistes are named the same. This could be modified were 2 Pistes named the same, by including another attribue and making a composite key. A candidate for this would be piesteName and length. This would be more likely unique. However, for the sake of this design we shall assume no 2 Pistes may be named the same.

\subsubsection{Lift}
\textbf{liftName -\textgreater  type, summit, rise, length, operating}
\\[0.2cm]
liftName implies all other attributes of the Lift. Assuming that no 2 lifts are named the same, we can accept liftName as an appropriate Primary Key. However, should it be that some Lifts are named the same, we could adapt this to make a composite key composed of liftName and length, giving the key a uniquely identifiable key. For this design however, we shall continue to assume no 2 lifts are named the same.

\subsection{Unnormalized Structure}
Based upon the sample data provided, the unnormalized structure of the Database is as follows:
\\[0.2cm]
\underline{pisteName}\newline
grade\newline
length\newline
fall\newline
liftName*\newline
open\newline
\\[0.2cm]
\underline{liftName}\newline
type\newline
summit\newline
rise\newline
length\newline
operating\newline

\section{Normalizing the Data}

To bring the Piste table into 1NF, we must remove the repeated groups of data (liftName). To do this, we create a new relation, "Connection". This relation contains the Primary Keys of both the Piste and Lifts, composing a composite key using the foreign keys of both. Resulting relations:
\subsubsection{Piste}
\begin{myitemize}
\item\underline{pisteName}
\item grade
\item length
\item fall
\item open
\end{myitemize}
\subsubsection{Lift}
\begin{myitemize}
\item\underline{liftName}
\item type
\item summit
\item rise
\item length
\item operating
\end{myitemize}
\subsubsection{Connection}
\begin{myitemize}
\item \underline{pisteName*} (from Piste)
\item \underline{liftName*} (from Lift)
\end{myitemize}

The data is now in 1NF. Lift was already in 1NF as it had no repeating groups of data. The data is also in Second Normal Form (2NF) as it has only one Primary Key, and each of the attributes of the relations relies solely on this single Primary Key and not partially on something else. Additionally to this, all non-key attributes rely upon the Primary Key to be determined, and so the data as is is already in Third Normal Form (3NF). This relational model is now ready to be put into a database.
\section{PostgreSQL}


\end{document}
